\documentclass[a4paper,12pt]{article}
\usepackage[utf8]{inputenc}
\usepackage[T1]{fontenc}
% \usepackage[spanish]{babel}
\usepackage[colorlinks, citecolor=blue]{hyperref}
\hypersetup{urlcolor=cyan, linkcolor=cyan}
\usepackage{anysize}
\usepackage{minted}
\usepackage[most]{tcolorbox}
\definecolor{lightgreen}{rgb}{0.56, 0.93, 0.56}
\definecolor{moonstoneblue}{rgb}{0.45, 0.66, 0.76}
\marginsize{25mm}{25mm}{25mm}{25mm}
\linespread{1.2}

\title{Brief introduction to Python}
\author{Daniel Maldonado}
\date{February 2022}

\begin{document}
{\scshape\bfseries \maketitle}

\tableofcontents

\newpage
\section{Introduction}

\subsection{Why MedPCPy}

Within behavior analysis an often found limiting factor is the inability to process data efficiently. Although given enough time most researchers would probably find ways to manage their data with more or less swiftness, many still struggle with manually converting their files to a more manageable format and making manual counts. Excel macros are a great tool for this, but they still require a great deal of engagement, and given that they require human interaction they are prone to errors.

The motivation for making this library was to provide a free and accesible tool that allowed both new and seasoned researchers to quickly gather the data they need without needing to spend time learning to code. This tool aims to be easy to learn and use, less error prone than regular scripting, and above all free (as in {\slshape gratis} and as in {\slshape libre}) for everyone. Med provides their own software for a similar purpose, but it is prohibitively costly. Those who develop open tools share the belief that one cannot put a price neither on knowledge nor on the tools to produce it.

\subsection{About this package}

The team behind {\scshape MedPCPy} has extensively worked to make sure that the library works in all conditions. Plenty of bugs and special cases have been found, and all have been fixed to the best of our abilities. However---as is the case with anything programming-related---something is bound to go wrong at some point. It is likely that more bugs will be found, either by us or by users. We encourage you to try and test the library in different cases, and to let us know when things go wrong so that we can improve the library and make it more useful for everyone.

If you find that something does not work as expected, or if you have any ideas on how to improve the library, please feel free to either open an issue on the project's \href{https://github.com/JuodaanViinaa/Laboratorio}{GitHub} page, or \href{mailto:maldonadodaniel96@outlook.com}{email us}.

\newpage
\section{Brief introduction to Python}




\end{document}
